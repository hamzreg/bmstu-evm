\chapter{Технологическая часть}

В данном разделе будут указаны средства реализации, будут представлены листинги кода, а также функциональные тесты.

\section{Средства реализации}

Реализация данной лабораторной работы выполнялась при помощи языка программирования Python \cite{python}. Выбор ЯП обусловлен простотой синтаксиса, большим числом библиотек и эффективностью визуализации данных.

Замеры времени проводились при помощи функции process\_time из библиотеки time \cite{python-time}.

\section{Сведения о модулях программы}

Программа состоит из следующих модулей:

\begin{itemize}
	\item main.py - главный файл программы, предоставляющий пользователю меню для выполнения основных функций;
	\item dictionary.py - файл, содержащий функции работы со словарями и методы поиска в словаре;
	\item test\_func.py - файл, содержащий функции сравнительного анализа указанных алгоритмов;
	\item graph\_result.py - файл, содержащий функции визуализации сравнительного анализа описанных алгоритмов.
\end{itemize}

\section{Листинги кода}

Поиск полным перебором приведен в листинге \ref{lst:brute-force}, бинарный поиск - в листнге \ref{lst:binary}, сегментный поиск - в листинге \ref{lst:segment}.

\begin{center}
\captionsetup{justification=raggedright,singlelinecheck=off}
\begin{lstlisting}[label=lst:brute-force,caption=Полный перебор]
def brute_force(self, need):

    compares = 0
    value = None

    keys = self.data.keys()

    for key in keys:
        compares += 1

        if key == need:
            value = self.data[key]
            return compares, value
        
    return Dictionary.NO_RESULT, value
\end{lstlisting} 
\end{center}
\begin{center}
\captionsetup{justification=raggedright,singlelinecheck=off}
\begin{lstlisting}[label=lst:binary,caption=Бинарный поиск]
def binary_search(self, need, sorted_data):
    compares = 0
    value = None

    keys = list(sorted_data)
    left = 0
    right = len(keys) - 1

    while left <= right:
        compares += 1
        middle = (left + right) // 2
        now_key = keys[middle]

        if now_key < need:
            left = middle + 1
        elif now_key > need:
            right = middle - 1
        else:
            value = sorted_data[now_key]
            return compares, value
        
    return Dictionary.NO_RESULT, value
\end{lstlisting}
\end{center}
\begin{center}
\captionsetup{justification=raggedright,singlelinecheck=off}
\begin{lstlisting}[label=lst:segment,caption=Сегментный поиск]
def segment_data(self):
    segments = {symbol: 0 for symbol in "ABCDEFGHIJMS12345679"}

    for key in self.data:
        segments[key[Dictionary.FIRST_SYMBOL]] += 1

    segments = self.sort_values(segments)

    segmented_data = {pair: dict() for pair in segments}

    for key in self.data:
        segmented_data[key[Dictionary.TITLE]].update({key: self.data[key]})
        
    return segmented_data

def segment_search(self, need, segmented_data):
    compares = 0
    value = None

    symbols = list(segmented_data.keys())

    for symbol in symbols:
        compares += 1

        if symbol == need[Dictionary.FIRST_SYMBOL]:
            segment_compares = 0

            for key in segmented_data[symbol]:
                segment_compares += 1

                if key == need:
                    value = segmented_data[symbol][key]
                    return compares + segment_compares, value
                
            return Dictionary.NO_RESULT, value

        
    return Dictionary.NO_RESULT, value
\end{lstlisting}
\end{center}

\section{Функциональные тесты}

В таблице \ref{tbl:func_test} приведены функциональные тесты для функций программы. Все тесты пройдены успешно.

\begin{center}
    \captionsetup{justification=raggedright,singlelinecheck=off}
    \begin{longtable}[c]{|c|c|c|c|c|}
    \caption{Функциональные тесты\label{tbl:func_test}} \\ \hline
		Ключ & Тип поиска & Ожидаемый результат \\
		\hline
		Пустой ввод &
		Любой &
		Сообщение об ошибке \\
		\hline
		A &
		Любой &
		Фильм не найден. \\
		\hline
		37 Seconds &
		Полный перебор &
		37 Seconds ---> TV-MA, 90 \\
		\hline
		Friends &
		Бинарный &
		Friends ---> TV-14, 11 \\
		\hline
		1983 &
		Сегментный &
		1983 ---> TV-MA, 19 \\
		\hline
	\end{longtable}
\end{center}

\section{Вывод}

Были реализованы функции алгоритмов поиска в словаре. Было проведено функциональное тестирование указанных функций.